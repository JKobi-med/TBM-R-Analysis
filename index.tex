% Options for packages loaded elsewhere
\PassOptionsToPackage{unicode}{hyperref}
\PassOptionsToPackage{hyphens}{url}
\PassOptionsToPackage{dvipsnames,svgnames,x11names}{xcolor}
%
\documentclass[
]{agujournal2019}

\usepackage{amsmath,amssymb}
\usepackage{iftex}
\ifPDFTeX
  \usepackage[T1]{fontenc}
  \usepackage[utf8]{inputenc}
  \usepackage{textcomp} % provide euro and other symbols
\else % if luatex or xetex
  \usepackage{unicode-math}
  \defaultfontfeatures{Scale=MatchLowercase}
  \defaultfontfeatures[\rmfamily]{Ligatures=TeX,Scale=1}
\fi
\usepackage{lmodern}
\ifPDFTeX\else  
    % xetex/luatex font selection
\fi
% Use upquote if available, for straight quotes in verbatim environments
\IfFileExists{upquote.sty}{\usepackage{upquote}}{}
\IfFileExists{microtype.sty}{% use microtype if available
  \usepackage[]{microtype}
  \UseMicrotypeSet[protrusion]{basicmath} % disable protrusion for tt fonts
}{}
\makeatletter
\@ifundefined{KOMAClassName}{% if non-KOMA class
  \IfFileExists{parskip.sty}{%
    \usepackage{parskip}
  }{% else
    \setlength{\parindent}{0pt}
    \setlength{\parskip}{6pt plus 2pt minus 1pt}}
}{% if KOMA class
  \KOMAoptions{parskip=half}}
\makeatother
\usepackage{xcolor}
\setlength{\emergencystretch}{3em} % prevent overfull lines
\setcounter{secnumdepth}{5}
% Make \paragraph and \subparagraph free-standing
\makeatletter
\ifx\paragraph\undefined\else
  \let\oldparagraph\paragraph
  \renewcommand{\paragraph}{
    \@ifstar
      \xxxParagraphStar
      \xxxParagraphNoStar
  }
  \newcommand{\xxxParagraphStar}[1]{\oldparagraph*{#1}\mbox{}}
  \newcommand{\xxxParagraphNoStar}[1]{\oldparagraph{#1}\mbox{}}
\fi
\ifx\subparagraph\undefined\else
  \let\oldsubparagraph\subparagraph
  \renewcommand{\subparagraph}{
    \@ifstar
      \xxxSubParagraphStar
      \xxxSubParagraphNoStar
  }
  \newcommand{\xxxSubParagraphStar}[1]{\oldsubparagraph*{#1}\mbox{}}
  \newcommand{\xxxSubParagraphNoStar}[1]{\oldsubparagraph{#1}\mbox{}}
\fi
\makeatother


\providecommand{\tightlist}{%
  \setlength{\itemsep}{0pt}\setlength{\parskip}{0pt}}\usepackage{longtable,booktabs,array}
\usepackage{calc} % for calculating minipage widths
% Correct order of tables after \paragraph or \subparagraph
\usepackage{etoolbox}
\makeatletter
\patchcmd\longtable{\par}{\if@noskipsec\mbox{}\fi\par}{}{}
\makeatother
% Allow footnotes in longtable head/foot
\IfFileExists{footnotehyper.sty}{\usepackage{footnotehyper}}{\usepackage{footnote}}
\makesavenoteenv{longtable}
\usepackage{graphicx}
\makeatletter
\def\maxwidth{\ifdim\Gin@nat@width>\linewidth\linewidth\else\Gin@nat@width\fi}
\def\maxheight{\ifdim\Gin@nat@height>\textheight\textheight\else\Gin@nat@height\fi}
\makeatother
% Scale images if necessary, so that they will not overflow the page
% margins by default, and it is still possible to overwrite the defaults
% using explicit options in \includegraphics[width, height, ...]{}
\setkeys{Gin}{width=\maxwidth,height=\maxheight,keepaspectratio}
% Set default figure placement to htbp
\makeatletter
\def\fps@figure{htbp}
\makeatother
% definitions for citeproc citations
\NewDocumentCommand\citeproctext{}{}
\NewDocumentCommand\citeproc{mm}{%
  \begingroup\def\citeproctext{#2}\cite{#1}\endgroup}
\makeatletter
 % allow citations to break across lines
 \let\@cite@ofmt\@firstofone
 % avoid brackets around text for \cite:
 \def\@biblabel#1{}
 \def\@cite#1#2{{#1\if@tempswa , #2\fi}}
\makeatother
\newlength{\cslhangindent}
\setlength{\cslhangindent}{1.5em}
\newlength{\csllabelwidth}
\setlength{\csllabelwidth}{3em}
\newenvironment{CSLReferences}[2] % #1 hanging-indent, #2 entry-spacing
 {\begin{list}{}{%
  \setlength{\itemindent}{0pt}
  \setlength{\leftmargin}{0pt}
  \setlength{\parsep}{0pt}
  % turn on hanging indent if param 1 is 1
  \ifodd #1
   \setlength{\leftmargin}{\cslhangindent}
   \setlength{\itemindent}{-1\cslhangindent}
  \fi
  % set entry spacing
  \setlength{\itemsep}{#2\baselineskip}}}
 {\end{list}}
\usepackage{calc}
\newcommand{\CSLBlock}[1]{\hfill\break\parbox[t]{\linewidth}{\strut\ignorespaces#1\strut}}
\newcommand{\CSLLeftMargin}[1]{\parbox[t]{\csllabelwidth}{\strut#1\strut}}
\newcommand{\CSLRightInline}[1]{\parbox[t]{\linewidth - \csllabelwidth}{\strut#1\strut}}
\newcommand{\CSLIndent}[1]{\hspace{\cslhangindent}#1}

\usepackage{url} %this package should fix any errors with URLs in refs.
\usepackage{lineno}
\usepackage[inline]{trackchanges} %for better track changes. finalnew option will compile document with changes incorporated.
\usepackage{soul}
\linenumbers
\makeatletter
\@ifpackageloaded{caption}{}{\usepackage{caption}}
\AtBeginDocument{%
\ifdefined\contentsname
  \renewcommand*\contentsname{Table of contents}
\else
  \newcommand\contentsname{Table of contents}
\fi
\ifdefined\listfigurename
  \renewcommand*\listfigurename{List of Figures}
\else
  \newcommand\listfigurename{List of Figures}
\fi
\ifdefined\listtablename
  \renewcommand*\listtablename{List of Tables}
\else
  \newcommand\listtablename{List of Tables}
\fi
\ifdefined\figurename
  \renewcommand*\figurename{Figure}
\else
  \newcommand\figurename{Figure}
\fi
\ifdefined\tablename
  \renewcommand*\tablename{Table}
\else
  \newcommand\tablename{Table}
\fi
}
\@ifpackageloaded{float}{}{\usepackage{float}}
\floatstyle{ruled}
\@ifundefined{c@chapter}{\newfloat{codelisting}{h}{lop}}{\newfloat{codelisting}{h}{lop}[chapter]}
\floatname{codelisting}{Listing}
\newcommand*\listoflistings{\listof{codelisting}{List of Listings}}
\makeatother
\makeatletter
\makeatother
\makeatletter
\@ifpackageloaded{caption}{}{\usepackage{caption}}
\@ifpackageloaded{subcaption}{}{\usepackage{subcaption}}
\makeatother

\ifLuaTeX
  \usepackage{selnolig}  % disable illegal ligatures
\fi
\usepackage{bookmark}

\IfFileExists{xurl.sty}{\usepackage{xurl}}{} % add URL line breaks if available
\urlstyle{same} % disable monospaced font for URLs
\hypersetup{
  pdftitle={TBM impact on methylation of neuroplasticity-associated genes (BDNF, Tau, PSD95)},
  pdfauthor={Jonas Janik Ralf Koberschinski; Martin Max Schumacher; Kirsten Jahn; Helge Frieling; Christopher Sinke; Kristin Jünemann; Clara E. James; Florian Worschech; Damien Marie; Eckart Altenmüller; Tillmann Horst Christoph Krüger},
  pdfkeywords={Piano, Aging, Neuroplasticity, Methylation, BDNF, Tau, PSD95},
  colorlinks=true,
  linkcolor={blue},
  filecolor={Maroon},
  citecolor={Blue},
  urlcolor={Blue},
  pdfcreator={LaTeX via pandoc}}


\journalname{Epigenetics Music}

\draftfalse

\begin{document}
\title{TBM impact on methylation of neuroplasticity-associated genes
(BDNF, Tau, PSD95)}

\authors{Jonas Janik Ralf Koberschinski\affil{1,2,3}, Martin Max
Schumacher\affil{4}, Kirsten Jahn\affil{5}, Helge
Frieling\affil{5}, Christopher Sinke\affil{3}, Kristin
Jünemann\affil{3}, Clara E. James\affil{6}, Florian
Worschech\affil{7,8}, Damien Marie\affil{6,9}, Eckart
Altenmüller\affil{7,8}, Tillmann Horst Christoph Krüger\affil{3,8}}
\affiliation{1}{Brandenburg Medical School, Neuruppin,
Germany, }\affiliation{2}{Laboratory of Molecular Neurosciences (LMN),
Dept. of clinical psychiatry, Hannover Medical School, Hannover,
Germany, }\affiliation{3}{Devision of Clinical Psychology and Sexual
Medicine, Dept. of Psychiatry, Social Psychiatry and Psychotherapy,
Hannover Medical School, Hannover,
Germany, }\affiliation{4}{Switzerland, }\affiliation{5}{Laboratory of
Molecular Neurosciences (LMN), Dept. of clinical psychiatry, Hannover
Medical School, Hannover Germany, }\affiliation{6}{Geneva Musical Minds
Lab, Geneva School of Health Sciences, University of Applied Sciences
and Arts Western Switzerland (HES-SO), Geneva,
Switzerland, }\affiliation{7}{Institute of Music Physiology and
Musicians′ Medicine, Hannover University of Music, Drama and Media,
Hannover, Germany, }\affiliation{8}{Center for Systems Neuroscience,
Hannover, Germany, }\affiliation{9}{Faculty of Psychology and
Educational Sciences,University of Geneva, Geneva, Switzerland, }
\correspondingauthor{Jonas Janik Ralf
Koberschinski}{jonas.koberschinski@mhb-fontane.de}


\begin{abstract}
\ldots{}
\end{abstract}

\section*{Plain Language Summary}
\ldots{}




\section{Introduction}\label{introduction}

\subsection{BDNF (Brain-Derived Neurotrophic
Factor)}\label{bdnf-brain-derived-neurotrophic-factor}

\subsubsection{Association with
Neuroplasticity}\label{association-with-neuroplasticity}

BDNF is a crucial neurotrophin that supports the survival,
differentiation, and growth of neurons. It plays a central role in
neuroplasticity by facilitating synaptic plasticity, which is the
ability of synapses to strengthen or weaken over time in response to
increases or decreases in their activity. It enhances long-term
potentiation (LTP), a process associated with learning and memory. BDNF
promotes synaptic plasticity by influencing dendritic growth and spine
density.

\subsubsection{Epigenetic Regulation}\label{epigenetic-regulation}

\paragraph{DNA Methylation}\label{dna-methylation}

The promoter region of the BDNF gene can be methylated, which generally
suppresses its expression. Methylation of BDNF can be influenced by
various factors, including environmental stimuli and stress.

\paragraph{Histone Modifications}\label{histone-modifications}

Post-translational modifications of histones can also regulate BDNF
expression. For example, acetylation of histones is associated with
increased BDNF expression.

\subsubsection{Transcription Factors}\label{transcription-factors}

Factors such as CREB (cAMP response element-binding protein) regulate
BDNF expression through binding to its promoter region.

\subsection{PSD95 (Postsynaptic Density Protein
95)}\label{psd95-postsynaptic-density-protein-95}

\subsubsection{Association with
Neuroplasticity}\label{association-with-neuroplasticity-1}

PSD95 is a scaffolding protein found in the postsynaptic density of
neurons. It is involved in organizing and stabilizing synaptic proteins,
and it plays a key role in the regulation of synaptic transmission and
plasticity. It interacts with glutamate receptors (e.g., NMDA and AMPA
receptors), thereby modulating synaptic strength and contributing to
synaptic plasticity mechanisms such as LTP and LTD (long-term
depression).

\subsubsection{Epigenetic Regulation}\label{epigenetic-regulation-1}

DNA Methylation and Histone Modifications: The expression of PSD95 can
be influenced by DNA methylation and histone acetylation. Epigenetic
changes can affect its expression levels and, consequently, synaptic
plasticity and cognitive functions.

\subsubsection{Activity-Dependent
Regulation}\label{activity-dependent-regulation}

Synaptic activity and neuronal signaling pathways can alter the
expression of PSD95 through epigenetic mechanisms, including the
regulation of transcription factors.

\subsection{Tau}\label{tau}

\subsubsection{Association with
Neuroplasticity}\label{association-with-neuroplasticity-2}

Tau is a microtubule-associated protein that stabilizes microtubules and
supports axonal transport. It plays a critical role in maintaining
neuronal structure and function. Dysregulation of Tau, such as
hyperphosphorylation, is associated with neurodegenerative diseases like
Alzheimer's disease. Tau pathology can disrupt neuroplasticity by
impairing synaptic function and neuronal integrity.

\subsubsection{Epigenetic Regulation}\label{epigenetic-regulation-2}

\paragraph{DNA Methylation and Histone
Modifications}\label{dna-methylation-and-histone-modifications}

Epigenetic regulation of Tau involves changes in DNA methylation
patterns and histone modifications. Abnormal epigenetic changes can
affect Tau expression and its pathological modifications.

\paragraph{Phosphorylation}\label{phosphorylation}

Tau's phosphorylation status is influenced by signaling pathways and can
affect its interactions with microtubules and its role in
neuroplasticity. Epigenetic Effects of Cognitive Activity (e.g., Making
Music)

\subsubsection{Methylation and Cognitive
Activity}\label{methylation-and-cognitive-activity}

\paragraph{Cognitive Activity and
Methylation:}\label{cognitive-activity-and-methylation}

Research indicates that cognitive activities, including complex
activities like making music, can influence DNA methylation patterns.
Such activities are associated with changes in gene expression related
to brain function and plasticity.

\paragraph{Making Music}\label{making-music}

Engaging in musical activities has been shown to affect the expression
of genes involved in neuroplasticity, including BDNF. Studies suggest
that musical training can lead to epigenetic changes that enhance
synaptic plasticity and cognitive functions.

\subsection{Relevant Publications}\label{relevant-publications}

\begin{itemize}
\tightlist
\item
  ``Epigenetic Regulation of BDNF in the Brain: Implications for
  Cognitive Function and Neuroplasticity'' - This review discusses the
  impact of epigenetic modifications on BDNF expression and its role in
  neuroplasticity and cognitive function.
\item
  ``Music Training and Brain Plasticity: Evidence from Functional and
  Structural Neuroimaging'' - This article reviews how musical training
  influences brain structure and function, including changes in
  epigenetic regulation.
\item
  ``The Impact of Cognitive and Physical Activities on DNA Methylation
  and Cognitive Function'' - This study explores how various cognitive
  activities, including music, influence DNA methylation and cognitive
  outcomes.
\item
  ``The Influence of Environmental Enrichment on Epigenetic
  Modifications: Implications for Learning and Memory'' - This paper
  discusses how environmental enrichment, which includes activities like
  music, affects epigenetic modifications related to learning and
  memory.
\end{itemize}

\subsection{Summary}\label{summary}

BDNF, PSD95, and Tau are crucial for neuroplasticity, influencing
synaptic function, and neuronal structure. Their expressions are
regulated by various epigenetic mechanisms, including DNA methylation
and histone modifications. Cognitive activities, such as making music,
have been shown to affect epigenetic regulation and potentially enhance
neuroplasticity. There is growing evidence supporting the idea that
engaging in complex cognitive activities can lead to beneficial
epigenetic changes that positively impact brain function.

\section{Methods}\label{sec-data-methods}

\subsection{Subjects}\label{subjects}

Study participants were recruited at Hannover Medical School, Dept. for
Clinical Psychiatry, Social Psychiatry and Psychotherapy, Division of
Clinical Psychology and Sexual Medicine and at the Geneva Musical Minds
Lab, Geneva School of Health Sciences, University of Applied Sciences
and Arts Western Switzerland (HES-SO), Geneva, Switzerland via newspaper
advertisement and placards at public places. Demographics are given in
the results in Table 1. Prerequisites for participation in the study
were an overall good health, being right-handed, retired, and
non-reliant on hearing-aids.

\subsection{Study design}\label{study-design}

The first time point was before the weekly practical piano lessons were
started (or theoretical music lessons without any practical exercises in
the control group, respectively), the second time point was after half a
year, and the third time point was one year after the first assessment
(for the long-term observation of potential beneficial effects of
piano-lessons) for every individual participant. The time points each
cover a period of several months, as all the participants could not be
examined on the same day, since MRI scans were also performed as part of
the original study (T0: March to June 2019, T1: August 2019 to February
2020, T2: June 2020 to November 2020). A small subgroup of participants
were recruited as early as in January 2019 (T0), and had their
consecutive examinations in August 2019 (T1) and in the end of
February/beginning of March 2020 (T2). In other words, their T2 was
directly before the lockdown started. Therefore, this small subgroup was
excluded from the main analyses.

\subsection{DNA isolation}\label{dna-isolation}

Genomic DNA (gDNA) for telomere measurements was isolated in the
Institute of Molecular and Translational Therapeutic Strategies of
Hannover Medical School according to standard procedures from 50 µL
blood using the DNeasy Blood \& Tissue Kit (Qiagen \#69506) and stored
at −20 °C. DNA samples were diluted in 96-well plates to a fixed
concentration of 10 ng/µl. The blood samples were received anonymized
and prior to gDNA isolation the order of the samples was further
randomized to minimize potential batch effects.

\subsection{Gene Amplification and
Sequencing}\label{gene-amplification-and-sequencing}

To investigate the methylation status of the target genes---BDNF, PSD95,
and Tau---specific regions of interest were amplified using polymerase
chain reaction (PCR) followed by Sanger sequencing. Genomic DNA (gDNA)
extracted from blood samples was bisulfite-converted using standard
procedures (Kit ergänzen), which converts unmethylated cytosines to
uracils while leaving methylated cytosines unchanged, enabling
differentiation between methylated and unmethylated CpG sites (Frommer
et al., 1992). PCR primers (Sequenz ergänzen) were designed to target
CpG-rich regions within the promoter or regulatory regions of each gene,
as these regions are often key in regulating gene expression through DNA
methylation. Amplifications were performed using high-fidelity DNA
polymerases to minimize errors, and the resulting PCR products were
verified for size and integrity via agarose gel electrophoresis (Clark
et al., 1994). Once verified, the products were purified using a
standard PCR purification kit. Sanger Sequencing was then performed on
the purified PCR products to identify the methylation patterns at
individual CpG sites. This method was chosen for its high accuracy in
detecting methylation at single-nucleotide resolution (Sanger et al.,
1977). Sequencing reactions were carried out using fluorescently labeled
dideoxynucleotides, and the resulting sequences were analyzed to
identify CpG methylation patterns.

\subsection{CpG Sites and Methylation
Analysis}\label{cpg-sites-and-methylation-analysis}

CpG sites---regions of DNA where a cytosine nucleotide is followed by a
guanine nucleotide---are often found in gene promoters and are key
regulators of gene expression. Methylation at these sites can silence
gene expression, making them critical points of study in understanding
gene regulation (Bird, 2002). For each of the three genes (BDNF, PSD95,
and Tau), multiple CpG sites were identified within the promoter or
other regulatory regions to assess their methylation status. By
examining these CpG islands, we aimed to determine whether the degree of
methylation differed between the intervention (piano lessons) and
control groups.

\subsection{Data Processing via ESME}\label{data-processing-via-esme}

The methylation status of individual CpG sites was quantified using the
Epigenetic Sequencing Methylation Analysis software (ESME), which is
specifically designed for analyzing bisulfite sequencing data (Lewin et
al., 2004). ESME processes bisulfite-converted Sanger sequencing data
and compares the cytosine-uracil conversion patterns to the reference
sequence to detect methylation at CpG sites. Specifically, ESME
calculates the percentage of methylation at each CpG site by determining
the ratio of methylated cytosines to total cytosines (methylated +
unmethylated) for each site. ESME allows for a high-resolution
quantification of methylation levels at each CpG site, providing decimal
values (e.g., 0.75 represents 75\% methylation at a specific CpG site).
The software also corrects for any potential sequencing errors and can
handle multiple sequencing runs to improve data accuracy. Each (not
each. Depending on the quality of the measurments some were ignored. The
dropped CpGs are mentioned in the Excel-titles and listed in .txt-files)
gene region's methylation data, broken down by CpG sites, was compiled
for statistical analysis (Lewin et al., 2004).

\subsection{Data Transfer and Further
Processing}\label{data-transfer-and-further-processing}

After methylation percentages were obtained from ESME, the data were
exported to Microsoft Excel for preliminary organization and data
structuring. In Excel, CpG site-specific methylation data were aligned
with participant identifiers, experimental groups (piano vs.~control),
and time points (T0, T1, T2). Data were thoroughly checked for
consistency, and any missing or potentially erroneous data points were
flagged for review. In this step, we created a structured dataset for
each gene, consisting of individual participants' methylation
percentages at each CpG site. The Excel file was formatted to ensure
seamless transfer into statistical software for further analysis. (The
Data was restructured in SPSS.)

\subsection{Data Structuring and Statistical
Analysis}\label{data-structuring-and-statistical-analysis}

The structured Excel data were subsequently imported into SPSS for
statistical analysis. In SPSS, the methylation data for each gene and
CpG site were organized into a comprehensive dataset, where
participants' CpG site methylation values were aligned with their
respective experimental group (piano or control) and time points (T0,
T1, and T2). To analyze the effects of piano lessons on the methylation
status of the BDNF, PSD95, and Tau genes, we employed generalized linear
models (GLMs). GLMs are flexible and suitable for modeling non-normally
distributed data, allowing for the inclusion of continuous, binary, and
categorical variables. This approach enabled us to examine changes in
methylation levels across the different time points (T0, T1, T2) while
controlling for potential covariates such as age and baseline
methylation levels. In the model, time was treated as a repeated
measure, and the group (piano vs.~control) was included as a fixed
factor. Methylation percentages at each CpG site (expressed as
continuous variables) were the dependent variables. We also included
interaction terms to assess whether the change in methylation over time
differed between the piano and control groups. The model used a logit
link function to handle the proportion-based methylation data, ensuring
the analysis accurately reflected the bounded nature of the dependent
variable (0 to 1). Model significance was assessed at p \textless{}
0.05, and Bonferroni corrections were applied to account for multiple
comparisons across the different CpG sites and genes. Model fit was
evaluated using the Akaike Information Criterion (AIC) and Bayesian
Information Criterion (BIC) to compare the fit of different models and
ensure robustness in detecting group differences over time (McCullagh \&
Nelder, 1989).

\section{Results}\label{results}

\subsection{General}\label{general}

\begin{longtable}[]{@{}lll@{}}
\toprule\noalign{}
\endhead
\bottomrule\noalign{}
\endlastfoot
~ & Piano Group & Control Group \\
Mean Age +/- SD (years) & 68.9 +/- 2.9 & 69.5 +/- 4 \\
Min./Max. & 64/76 & 62/78 \\
N & 29??? & 26??? \\
Female/Male (\%) & 44.8/55.2 & 38.5/61.5 \\
\end{longtable}

Table 1: Demographics of the study cohort

\subsection{Methylation Levels}\label{methylation-levels}

\begin{itemize}
\tightlist
\item
  Die Gene sind allgemein eher gering methyliert (Mean \textless{} 0,08,
  Median \textless{} 0,04). \textbf{Hohe Expressionsrate!}
\item
  Dabei ist PSD95 am geringsten und BDNF am höchsten metyhyliert.
\item
  Bei Tau sind die CpGs promoterfern mehr meyhtliert - bei PSD95
  umgekehrt (\emph{Boxplots mit CpG-Methylierung der Gene!})
\item
  \textbf{Lagemaß u. Streuungsmaß?}
\end{itemize}

\subsubsection{Gruppendifferenz}\label{gruppendifferenz}

\begin{itemize}
\tightlist
\item
  Bei BDNF zeigt sich eine leichte Differenz zwischen der Kontrollgruppe
  (Median: 0.07, Mean: 0.07326) und der Pianogruppe (Median: 0.064,
  Mean: 0.06598), welche eine niedrigere Metyhlierung (Median: -0.006,
  Mean: -0.00728) aufwies. \textbf{Streuungsmaße?}
\item
  Weiterhin zeigen sich für die arithmetischen Mittel von BDNF und PSD95
  zwei Modi.
\item
  Tau war unimodal.
\end{itemize}

\subsubsection{Zeitdifferenz}\label{zeitdifferenz}

\begin{itemize}
\tightlist
\item
  Bei BDNF zeigt sich, dass sich die Mittelwerte (Arithmetisch und
  Median) von T0 zu T1 leicht absenken und zu T2 wieder über das Niveau
  von T0 ansteigen.
\item
  Bei Tau steigt das arithmetische Mittel der Methylierung leicht von T0
  zu T1 und hält das Niveau bei T2.
\item
  Bei PSD95 verbleibt der Median der Methylierung gleichbleibend bei 0,
  während das arithmetische Mittel der Methylierung von T0 zu T1 steigt
  aber bei T2 wieder auf das Ausgangsniveau zurückfällt.
\end{itemize}

\subsection{Tests und Modelle}\label{tests-und-modelle}

\section{Discussion}\label{discussion}

\subsection{Limitations}\label{limitations}

\section{Conclusion}\label{conclusion}

\section{Declarations}\label{declarations}

\subsection{Ethics approval and consent to
participate}\label{ethics-approval-and-consent-to-participate}

The study protocol was reviewed and approved by the Hannover Medical
School ethics committee (approval number 3604-2017). All study
participants gave their written informed consent to participate in the
study.

\subsection{Consent for publication}\label{consent-for-publication}

Not applicable.

\subsection{Availability of data and
materials}\label{availability-of-data-and-materials}

Data are available from the corresponding author upon request via
jonas.koberschinski@mhb-fontane.de and via
https://github.com/JKobi-med/Promotion.

\subsection{Competing interests}\label{competing-interests}

The authors have no competing interests to declare.

\subsection{Funding}\label{funding}

This work was funded by the Swiss National Science Foundation (grant no.
100019E-170410 to Clara E. James) and the German Research Foundation
(grant no. 323965454 to Eckart Altenmüller). Financial support in
Switzerland was also provided by the Med. Kurt Fries Foundation, the
Dalle Molle Foundation, and the Edith Maryon Foundation (to Clara E.
James).

\subsection{Author Contributions}\label{author-contributions}

Conceptualization: T.H.C.K.\\
Methodology: K.Jü., C.S., S.C., K.Ja., J.K.\\
Software: S.C., C.S., K.Jü., K.Ja., J.K.\\
Formal Analysis: S.C., K.Ja., C.S. Investigation: S.C., C.B., K.Jü.\\
Resources: T.H.C.K., C.B., S.B., H.F.\\
Writing -- original draft preparation: J.K.\\
Writing -- review and editing: S.C., C.S., K. Jü., H.F., S.B., C. B.,
T.H.C.K.\\
Visualization, K.Ja., J.K.\\
Supervision, T.H.C.K., C.B., H.F., S.B.\\
All authors have read and agreed to the published version of the
manuscript.

\subsection{Acknowledgement}\label{acknowledgement}

The first author would like to thank Martin Max Schumacher for his
expert assistance and kind support.

\section*{References}\label{references}
\addcontentsline{toc}{section}{References}

\phantomsection\label{refs}
\begin{CSLReferences}{0}{1}
\vspace{1em}

\end{CSLReferences}




\end{document}
